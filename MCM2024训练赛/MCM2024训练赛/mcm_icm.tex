\documentclass[12pt]{article}
\usepackage{geometry}
\geometry{left=1in,right=0.75in,top=1in,bottom=1in}

%%%%%%%%%%%%%%%%%%%%%%%%%%%%%%%%%%%%%%%%
% Replace ABCDEF in the next line with your chosen problem
% and replace 1111111 with your Team Control Number
\newcommand{\Problem}{B}
\newcommand{\Team}{223}
%%%%%%%%%%%%%%%%%%%%%%%%%%%%%%%%%%%%%%%%
\usepackage{newtxtext}
\usepackage{hyperref}
\usepackage{amsmath,amssymb,amsthm}
\usepackage{lipsum}
\usepackage{booktabs}
\usepackage{float}
\usepackage{subfigure}
\usepackage[pdftex]{graphicx}
\usepackage{xcolor}
\usepackage{fancyhdr}
\usepackage{url}
\DeclareUnicodeCharacter{0300}{\'{a}}
\lhead{Team \Team}
\rhead{}
\cfoot{}
\newtheorem{theorem}{Theorem}
\newtheorem{corollary}[theorem]{Corollary}
\newtheorem{lemma}[theorem]{Lemma}
\newtheorem{definition}{Definition}
%%%%%%%%%%%%%%%%%%%%%%%%%%%%%%%%
\begin{document}
\graphicspath{{.}}  % Place your graphic files in the same directory as your main document
\DeclareGraphicsExtensions{.pdf, .jpg, .tif, .png}
\thispagestyle{empty}
\vspace*{-16ex}
\centerline{\begin{tabular}{*3{c}}
	\parbox[t]{0.3\linewidth}{\begin{center}\textbf{Problem Chosen}\\ \Large \textcolor{red}{\Problem}\end{center}}
	& \parbox[t]{0.3\linewidth}{\begin{center}\textbf{2024\\ MCM/ICM\\ Summary Sheet}\end{center}}
	& \parbox[t]{0.3\linewidth}{\begin{center}\textbf{Team Control Number}\\ \Large \textcolor{red}{\Team}\end{center}}	\\
	\hline
\end{tabular}}
%%%%%%%%%%% Begin Summary %%%%%%%%%%%
%标题
\begin{center}
	\Huge {Enhancing Taxi Route Planning: A Comprehensive Analysis of ISPL }\\
	\vspace{0.4cm}
	\normalsize\textbf{Summary}
\end{center}
\vspace{0.2cm}

\indent Taxi companies need to consider user preferences reflected in historical path data because the best route during a taxi ride may not necessarily be the shortest route. It is important to take into account both the distance and user preferences in order to determine the optimal route. Assigning weights to paths based on ambiguous user preferences and planning the shortest path under corresponding conditions belongs to a type of an inverse shortest path length problem(ISPL). To address this problem, we introduce edge PageRank to measure road weights and utilize simulated annealing algorithm to find the maximum SIM value.\\
\indent For the first task, we prove by contradiction that the SIM value in example 1 cannot be equal to 1. Additionally, we attempted to assign weights to each edge, \textbf{resulting in a SIM value of 0.75}.\\
\indent For the second task, we first preprocessed the data in Case 1. We placed the data points on a map of Beijing to observe their distribution characteristics and removed duplicate edges and identified edges that did not appear in any of the paths. We then investigated the number of times each edge was traversed in the paths and analyzed the distribution characteristics of the data.Then, we used \textbf{Dijkstra's algorithm} to find the shortest path length for each path. We sorted the paths in ascending order based on their starting and ending points. By processing all paths with the same starting point at once using an array, we significantly reduced the time complexity. Finally, we calculated that \textbf{the SIM value for Case 1 is 0.248971}.\\
\indent For the third task, in our initial model, we used \textbf{road length, road importance (measured by edge PageRank), and straight-line distance} to measure road weights. We calculated the straight-line distance based on the latitude and longitude of the starting and ending points. We calculated \textbf{the maximum SIM value to be 0.273278}. We then conducted further analysis and found that the straight-line distance had little impact on the SIM value. However, we discovered a significant relationship \textbf{between the data point's index and its location}. Therefore, we decided to optimize our model by excluding the straight-line distance, introduceing the edge index into the calculation formula for edge PageRank and adding a constant. Ultimately, we determined \textbf{several parameters, denoted as 26.27088, 0.01172792, and 0.256358}, to calculate the edge weights. It results in \textbf{a maximum SIM value of 0.298223}.\\
\indent For the fourth task,we first compared the optimization rates of the two models in terms of SIM values obtained compared to using length as edge weights. Next, we plotted scatter plots \textbf{comparing the distribution differences between edge weights and edge lengths}, and analyzed the impact of introducing PageRank into the weight calculation.\\
\indent Finally, we conducted \textbf{sensitivity tests} on the resulting model. We selected the parameters \textbf{strides and initial and final temperatures} for testing and found that the model exhibited good stability for different values of these parameters.\\
%关键词
\vspace{0.4cm}
 \textbf{Keywords: }Route Planning\quad Inverse Shortest Path Length \quad Simulated Annealing Algorithm
%%%%%%%%%%% End Summary %%%%%%%%%%%

%%%%%%%%%%%%%%%%%%%%%%%%%%%%%%
\clearpage
\pagestyle{fancy}
% Uncomment the next line to generate a Table of Contents
%\tableofcontents 
\newpage
\setcounter{page}{2}
\rhead{Page \thepage~of~25}
%%%%%%%%%%%%%%%%%%%%%%%%%%%%%%

%目录
\tableofcontents
\newpage

%正文部分
\section{Introduction}
\subsection{Problem Background}
\subsection{Restatement of the Task}
\subsection{Literature Review}
\subsection{Our Work}
\section{Assumptions and Justifications}
\subsection{Assumptions}
we would like to give the following assumptions://
\begin{itemize}
\item first
\item second
\end{itemize}
\subsection{Justifications}

\section{Notations}
\begin{table}[H]
	\begin{center}
		\begin{tabular}{cc}
			\toprule[1.5pt]
			Symbol&Definition\\
			\midrule[1pt]
			\(SIM\)&Value of the SIM\\
			\(a_i\)&Evaluation indicator\\
			\(\alpha\)& Sum of the shortest paths\\
			\(A\)& Sum of all paths\\
			\({u}\)&The starting point of the edge\\
			\({v}\)&The endpoint of the edge\\
			\({PR_e^{\gamma}(G)}\)&The PageRank of the edge e in graph G\\
			\(dist_e\)&The straight-line distance of the edge e\\
			\(length_e\)&The length of the edge e\\
			\(\omega_e\)&The weight of the edge e\\
			\bottomrule[1.5pt]
		\end{tabular}
	\end{center}
\end{table}
\section{Model}
\lipsum[1-4]





\section{Testing}
\indent\indent We have a model based on the simulated annealing algorithm, which is influenced by various parameters such as the maximum value of parameter variation (strides) and initial and final temperatures. We will conduct stability tests on these two parameters to evaluate the stability of the model.\\
\indent First, we keep the initial and final temperatures and parameter values unchanged in the algorithm. We adjust the strides to be 0.3 and 0.6, respectively, and run the program. The final results are as follows:
\begin{figure}[H]
    \centering
    \includegraphics[width=10cm,height=7.5cm]{不同步长.png}
    \caption{SIM with Different Strides}
\end{figure}
\indent We can see that the group with a smaller step size reached the optimum value first and then remained relatively stable. On the other hand, the group with a larger step size took longer to reach the optimum value. The maximum SIM values obtained from the two groups with different step sizes are close, indicating that the model is not sensitive to this parameter.\\
\indent Next, we kept the initial values and strides of all parameters unchanged and only adjusted the initial and final temperatures of the algorithm to be (50,1) and (25,0.5) with a purpose of controlling the rounds. The results are as follows:
\begin{figure}[H]
    \centering
    \includegraphics[width=8cm,height=6cm]{T.png}
    \caption{SIM with Different Initial and Final Temperatures}
\end{figure}
\indent We found that the group with a higher initial temperature calculated relatively higher results, while the group with a lower temperature had a lower probability of accepting smaller values, making it easier to reject smaller values and resulting in relatively lower results. However, there was no significant difference between the two groups, indicating a stable performance.\\
\indent To better showcase the tests we conducted, we have created the following table for comparison:
\begin{table}[H]%绘制结果权值表
    \begin{center}
        \caption{Robustness Test}{\vspace{0.5cm}}
        \begin{tabular}{cc}
        \hline
        \text{Trial}&\text{$SIM_{max}$}\\
        \hline
        \text{$T_0$=50  $T_{end}$=1 strides=0.3}&0.296183\\
        \hline
        \text{\quad $T_0$=25 $T_{end}$=0.5 strides=0.3}&0.293384\\       
        \hline
        \text{$T_0$=30  $T_{end}$=1 strides=0.6}&0.297507\\  
        \hline
        \text{$T_0$=30  $T_{end}$=1 strides=0.3}&0.298223\\       
        \hline
        \end{tabular}
    \end{center}
    \end{table}
\indent Considering the two indicators above, we believe that \textbf{our model exhibits strong stability and is not sensitive to changes in parameters}.


\section{Strengths and Weaknesses}
\subsection{Strengths}
\begin{itemize}
    \item For task 2, we utilized the heap-optimized Dijkstra's algorithm to calculate the shortest paths. We categorized the paths based on the starting point and computed them, which allowed us to achieve extremely fast computational speed within limited space complexity.
    \item For task 3, we employed PageRank, which takes into account the relationships between nodes in terms of topology, geographic location, and the importance of edges. The model we used is computationally simple and highly scalable. Additionally, by utilizing a simulated annealing algorithm, we achieved fast convergence and a strong ability to escape local maxima.
\end{itemize}
%%%%%%%%%%%%%%%%%%%%%%%%%%%%%%%%%%%%%%%%%%%%%%%%%%%%%%%%%%%%%%%%%%%%%%%%%%%%%%%%%%%%
\subsection{Weaknesses}
\begin{itemize}
    \item Due to the limited amount of actual data, the generalizability of the model has not been thoroughly validated.
    \item Due to the difficulty of convergence in high-dimensional spaces for simulated annealing algorithms, our model structure is relatively simple and has relatively weak expressiveness. Strong computational support is required to improve performance in higher dimensions.
    \item Due to limited computational performance and time constraints, the number of iterations for simulated annealing was insufficient, and the solution is still slightly away from the optimal solution.
\end{itemize}

\clearpage
\phantomsection
\addcontentsline{toc}{section}{Renference}
\tolerance=500
\begin{thebibliography}{99}
	\bibitem{1}Page L, Brin S, Motwani R, et al. The pagerank citation ranking: Bring order to the web[R]. Technical report, stanford University, 1998.
	\bibitem{2}Dijkstra E W. A note on two problems in connexion with graphs[M]//Edsger Wybe Dijkstra: His Life, Work, and Legacy. 2022: 287-290.
	\bibitem{3}Niu G, Cartoixà X, Grossi A, et al. Mechanism of the Key Impact of Residual Carbon Content on the Reliability of Integrated Resistive Random Access Memory Arrays[J]. The Journal of Physical Chemistry C, 2017, 121(12): 7005-7014.
	\bibitem{4}Leitão A, Vinhas A, Machado P, et al. A Genetic Algorithms Approach for Inverse Shortest Path Length Problems[J]. International Journal of Natural Computing Research (IJNCR), 2014, 4(4): 36-54.
	\bibitem{5}Steinbrunn M ,Moerkotte G, Kemper A. Heuristic and Randomized Optimization for the Join Ordering Problem[J ] . The VLDB Journal , 1997 , 6 (3) :8 - 17.
	\bibitem{6}Cui T, Hochbaum D S. Complexity of some inverse shortest path lengths problems[J]. Networks, 2010, 56(1): 20-29.
\end{thebibliography}
\appendix{\textbf{Appendix}\Huge}
\begin{verbatim}
    inline void Dijkstra(int s, int* d, int numv, vector<Edge>* adj, unordered_map<pair<int, int>, Edge, pair_hash>& map)
    {
        priority_queue<pair<int, int>, vector<pair<int, int>>, greater<pair<int, int>>> q;
        bool visited[maxV];
        for (int i = 0; i < numv; i++)
            d[i] = maxD;
        memset(visited, 0, sizeof(visited));
        d[s] = 0;
        int mind = maxD;
        int now = 0;
        q.push({ 0, s });
        while (true) {
            mind = maxD;
            if (q.empty())
                return;
            auto x = q.top();
            q.pop();
            int now = x.second, mind = x.first;
            if (visited[now])
                continue;
            visited[now] = 1;
            for (int i = 0; i < adj[now].size(); i++) {
                int to = adj[now][i].end;
                if (d[now] + map[{now, to}].weight < d[to])
                {
                    d[to] = d[now] + map[{now, to}].weight;
                    q.push({ d[to], to });
                }
            }
        }
    }
\end{verbatim}
\begin{verbatim}
    static double SIM(unordered_map<pair<int, int>, Edge, pair_hash>& map, vector<Edge>* edges) {
	ifstream file("path.txt");//data-processed
	string line;
	int simcnt = 0, cnt = 0, start0 = -1;
	int d[maxV];
	while (getline(file, line)) {
		istringstream ss(line);

		int pathlength = 0;
		int start, end;
		ss >> start;
		if (start0 != start) {
			dijkstra(start, d, numv, edges, map);
			start0 = start;
		}
		while (ss >> end) {
			pathlength += map[{start, end}].weight;
			start = end;
		}
		if (pathlength == d[end]) {
			simcnt++;
		}
		cnt++;
	}

	file.close();
	return 1.0 * simcnt / cnt;
}

\end{verbatim}

\begin{verbatim}
    static void Page_Rank(vector<Edge>* adj, vector<Edge>* antiadj, int numv, double* b, double a) {
	vector <Edge>new_Adj[maxV];
	vector <Edge>new_Antiadj[maxV];
	for (int i = 0; i < numv; i++) {
		double pr = 0;
		for (int k = 0; k < antiadj[i].size(); k++)
			pr += antiadj[i][k].weight;
		pr *= a;
		pr += b[i];
		pr /= adj[i].size();
		for (int j = 0; j < adj[i].size(); j++)
		{
			add_Edge(i, adj[i][j].end, pr, new_Adj);
			add_Edge(adj[i][j].end, i, pr, new_Antiadj);
		}
	}
	for (int i = 0; i < numv; i++)
	{
		for (int j = 0; j < adj[i].size(); j++)
			adj[i][j] = new_Adj[i][j];
		for (int j = 0; j < antiadj[i].size(); j++)
			antiadj[i][j] = new_Antiadj[i][j];
	}
}
\end{verbatim}

\begin{verbatim}
    struct pa {
	double a, b, c;
};
static pa SA(pa now, unordered_map<pair<int, int>, Edge, pair_hash>& pr, unordered_map<pair<int, int>, Edge, pair_hash>& length, vector<Edge>* edges, vector<Point> v) {
	unordered_map<pair<int, int>, Edge, pair_hash> weight;
	cal_weight(now, weight, pr, length, v);
	double sim1, sim2, dsim;
	sim1 = SIM(weight, edges);
	double T = 25, T_end = 0.5;
	while (T > T_end) {
		pa par = update(now);
		cal_weight(par, weight, pr, length, v);
		sim2 = SIM(weight, edges);

		dsim = (sim2 - sim1) * 10000;
			double r;
			r = (double)rand() / RAND_MAX;
			if (r < exp(dsim / T)) {
				now = par;
				sim1 = sim2;
			}
		}
		else {
			now = par;
			sim1 = sim2;
		}
		T *= 0.9;
	}
	myfile.close();
	return now;
}
\end{verbatim}
   





%%%%%%%%%%%%%%%%%%%%%%%%%%%%%%
\end{document}
\end