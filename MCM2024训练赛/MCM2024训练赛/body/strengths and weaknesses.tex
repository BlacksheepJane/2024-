\section{Strengths and Weaknesses}
\subsection{Strengths}
\begin{itemize}
    \item For task 2, we utilized the heap-optimized Dijkstra's algorithm to calculate the shortest paths. We categorized the paths based on the starting point and computed them, which allowed us to achieve extremely fast computational speed within limited space complexity.
    \item For task 3, we employed PageRank, which takes into account the relationships between nodes in terms of topology, geographic location, and the importance of edges. The model we used is computationally simple and highly scalable. Additionally, by utilizing a simulated annealing algorithm, we achieved fast convergence and a strong ability to escape local maxima.
\end{itemize}
%%%%%%%%%%%%%%%%%%%%%%%%%%%%%%%%%%%%%%%%%%%%%%%%%%%%%%%%%%%%%%%%%%%%%%%%%%%%%%%%%%%%
\subsection{Weaknesses}
\begin{itemize}
    \item Due to the limited amount of actual data, the generalizability of the model has not been thoroughly validated.
    \item Due to the difficulty of convergence in high-dimensional spaces for simulated annealing algorithms, our model structure is relatively simple and has relatively weak expressiveness. Strong computational support is required to improve performance in higher dimensions.
    \item Due to limited computational performance and time constraints, the number of iterations for simulated annealing was insufficient, and the solution is still slightly away from the optimal solution.
\end{itemize}
